\section*{Executive Summary}
\addcontentsline{toc}{section}{\protect\hphantom{\numberline{\thesection}}Executive Summary}

This report presents the preliminary design and analysis for a long range, Vertical Take Off and Landing (VTOL), Unmanned Aerial Vehicle (UAV). Through discussion with the Country Fire Service (CFS) the need for a VTOL UAV, to conduct bushfire scar scanning was identified. A detailed set of project objectives was developed from these discussions, through which specific system requirements were identified and a worse case mission profile developed. Based on quantitative analysis and understanding of fundamental aerodynamic principals these designs were ranked according to how well they fulfilled the project objectives. Two subsequent rounds of concept generation were undertaken culminating in two designs which were extensively evaluated through conducting preliminary sizing and determining required power consumption for completion of the worst case mission profile. From this a design was identified with a required battery weight of 3.185 kg. The final design is based upon a canard configuration which utilises a tilt-wing mechanism and modular wings to allow for continuing analysis in optimum propulsion system. The tilt-wing mechanism is designed to achieve an angular wing rotation of approximately 140 degrees and capacity to withstand 2.51 Nm of torque. The selected camera payload system was the FireFlight 3000, comprised of 2 Infra-Red (IR) Cameras and a Red Green Blue (RGB) camera, which performs on board monitoring and analysis of the bushfire area. At the allowable operating height of 400ft, stipulated by CASA regulations, this payload provides a spatial resolution of {0.2m$^2$}, sufficient for the monitoring applications identified. A prototype has been developed to ascertain the control complexities of the proposed device. The selected Programmable Flight controller for a low end prototype system is the Matek Systems F405-STD. This flight controller will integrate with preliminary power electronics to form a minimum complexity VTOL system to explore control feasibility. A preliminary system was developed for the computer vision landing site detection using sample images with proven success. \\

High level risk analysis reveled the potential that stakeholder expectations of innovation may not be met, due to tight timelines imposed on the design and research stages as a result of meeting workshop manufacturing deadlines. To adhere to requirements of both industrial and academic stakeholders, two desired project outcomes were identified. A Minimal Viable Product (MVP) containing elements of the systems described above, will be manufactured following a simple prototype stage intended to confirm projected aircraft control capabilities. The MVP is necessary for physical testing and validation of the design analysis and methodology thus satisfying academic requirements. In parallel, continued research will be conducted in optimising the overall design and investigating opportunities to improve flight efficiencies, fulfilling industrial requirements. Provided the MVP verifies the theoretical analysis, increased confidence may be placed on the optimised design, with increased complexity, to perform as expected. Project management projections expect the project to conclude within the required time frame, and within the project budget. 
