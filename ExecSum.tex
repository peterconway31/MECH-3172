\section*{Executive Summary}
\addcontentsline{toc}{section}{\protect\hphantom{\numberline{\thesection}}Executive Summary}

This report presents the preliminary design and analysis for a long range, Vertical Take Off and Landing (VTOL), Unmanned Aerial Vehicle (UAV). Through discussion with the Country Fire Service (CFS) the need for a VTOL UAV, to conduct bushfire scar scanning was identified. A detailed set of project objectives was developed from these discussions, through which specific system requirements were identified and a worse case mission profile developed. Based on quantitative analysis and understanding of fundamental aerodynamic principals these designs were ranked according to how well they fulfilled the project objectives. Two subsequent rounds of concept generation were undertaken culminating in two designs which were extensively evaluated through conducting preliminary sizing and determining required power consumption for completion of the worst case mission profile. \note{Rhys}{Include sentence about our split manufacturing/analysis approach}. From this a design was identified with a required battery weight of XXXX. The final design is based upon a canard configuration which utilises a tilt wing mechanism and modular wings to allow for continuing analysis in optimum propulsion system. The tilt wing mechanism is designed to achieve an angular wing rotation of approximately 140 degrees and capacity to withstand XXXX N of torque. The selected camera payload system was the FireFlight 3000, comprised of 2 Infra-Red (IR) Cameras and a Red Green Blue (RGB) camera, which performs on board monitoring and analysis of the bushfire area. At the allowable operating height of 400ft, stipulated by CASA regulations, this payload provides a spatial resolution of {0.2m$^2$}, sufficient for the monitoring applications identified. A prototype has been developed to ascertain the control complexities of the proposed device. The selected Programmable Flight controller for a low end prototype system is the Matek Systems F405-STD. This flight controller will integrate with preliminary power electronics to form a minimum complexity VTOL system to explore control feasibility. A preliminary system was developed for the computer vision landing site detection using sample images with proven success. 
\note{Sam}{Include sentance on overal risk mitigtion, budget, and expected outcomes. Validation and verificaiton of analysis -> scale up complexity, Verificaiton of anlaysis is performed through testing of a manufactured minimul viable product which will instill confidence that the optimsation design will also be valid}
